\documentclass{article}
\usepackage{multicol}
\usepackage[margin=1in]{geometry} % Adjust margins as needed
\PassOptionsToPackage{hyphens}{url}\usepackage{hyperref}
\begin{document}

\begin{multicols}{2}
\section{The Arduno Microcontroller}
The Arduino is a very popular microcontroller, that is often used to teach the basics of programming at schools. The programming language used for the Arduino is almost always C, which is a compiled, low-leval programming language. This means that is is very fast and uses a small amount of memory. The drawback is, that C is more difficult to learn than higher-level programming languages like Python. The difference between high- and low-level programming languages is the amount of control the user has over system. This can be a good thing for very performance critical applications, but high-level programming languages are generally easier. For example in Python, it is very easy to create a List and to insert or remove values. When attempting this in C, you quickly encounter pointers and otehr low-level concepts that can be very difficult for beginners.

\section{Problem}
This raises the question why you wouldn't just use Python on the Arduino to teach programming. The Problem is, that the Arduino Uno only has 2 kb of Ram and 32 kb of program memory \cite{Q1}. The Python interpreter alone is over 30 mb big and it links into an even bigger standard libary. Running a Python executable requires a much faster CPU and orders of magnitude more RAM than the Arduino UNO has to offer. So, at first glance, running Python on an Arduino seems impossible. But there is a project that attempts to solve this problem; MicroPython is an implementation of Python for microcontrollers, designed to use as few resources as possible. MicroPython is able to run a minimal subset of python with as little as 256 kb of flash nad 16kib of ram. But the Arduino Uno still only has an eigth of the capacity required to run MicroPython. The reason, why Python requires so much resources, is the way that it works different than C. To run a .c file, the C compiler translates it into machine code, which can then be executed. A python file on the other hand is run by the Python interpreter, which runs a python file line by line. This is a lot slower than a compiled C executable, but it enables a lot of abstractions which make Python an easier, higher-level language than C.



\begin{thebibliography}{99}
\bibitem{Q1} 
\url{https://docs.arduino.cc/hardware/uno-rev3/#tech-specs}
\bibitem{Q2}
\url{https://github.com/micropython/micropython/blob/master/README.md}
\end{thebibliography}
\end{multicols}

\end{document}


