\documentclass{article}
\usepackage{multicol}
\usepackage[margin=1in]{geometry} % Adjust margins as needed
\PassOptionsToPackage{hyphens}{url}\usepackage{hyperref}
\begin{document}

\begin{multicols}{2}

\section{Problem}
My project started two years ago, when a friend of mine told me, that I wouldn't be able to run Python on an Arduino microcontroller. I accepted the bet and started developing Pyduino. At first, one might think that running Python on an Arduino ist not very difficult, but catch is, that the Arduino Uno only has 2 kb of Ram and 32 kb of program memory \cite{Q1}. The Python executable alone is over 30 mb and it links into an even bigger standard libary. Running a Python executable requires a much faster CPU and orders of magnitude more RAM than the Arduino UNO has to offer. So, at first glance, running Python on an Arduino seems impossible. But there is a project that attempts to solve this problem; MicroPython is an implementation of Python for microcontrollers, designed to use as few resources as possible. MicroPython is able to run a minimal subset of python with as little as 256 kb of flash nad 16kib of ram. But the Arduino Uno still only has an eigth of the capacity required to run MicroPython.
\section{How Python works}
The Arduno is usually programmed with the C programming language, which works very well. This raises the question, what makes C so much bedder for the development of embedded systems than Python? The answer lies in the way that C Programs are executed. A .c File is compiled to executable machine code by a Compiler (like gcc or llvm). With some modifictaions, this executable file can then be run on a PC or a microcontroller. A .py File on the other hand is not ompiled, it is executed by an interpreter. The Python interpreter is a program that takes a Python file and runs it line by line. This results in slower execution, compared to C, but it enables some cool features, like for example dynamic Types.
\section{Python vs C}
This raises the question, why you couldn't just use C instead of Python to program an Arduno. To answer this, we have to consider the typical usecase of the Arduino microcontroller, which is mostly teaching the basics of Programming to students.



\begin{thebibliography}{99}
\bibitem{Q1} 
\url{https://docs.arduino.cc/hardware/uno-rev3/#tech-specs}
\bibitem{Q2}
\url{https://github.com/micropython/micropython/blob/master/README.md}
\end{thebibliography}
\end{multicols}

\end{document}


